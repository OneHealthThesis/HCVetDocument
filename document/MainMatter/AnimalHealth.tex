\chapter{Bienestar Animal. Historia Clínica Veterianria}\label{chapter:animalHealth}
El bienestar animal es un concepto amplio que incluye diversos elementos que contribuyen a la calidad de vida de un animal, incluidos los referidos en las “cinco necesidades” (libertades): necesidad de no sufrir hambre, sed ni desnutrición; necesidad de no experimentar miedo ni angustia; necesidad de vivir libre de incomodidad física y térmica; necesidad de no sufrir de dolor, lesiones y enfermedad; y necesidad de expresar patrones normales de comportamiento. 
Ha sido definido por la Organización Mundial de Sanidad Animal (OIE) como el término que describe la manera en que los individuos se enfrentan con el medio ambiente y que incluye su sanidad, sus percepciones, su estado anímico y otros efectos positivos o negativos que influyen sobre los mecanismos físicos y psíquicos del animal. (Manteca; 2012)
Por otra parte, el biólogo y profesor de bienestar animal Donald Broom lo describe como: “el equilibrio del estado físico y psicológico de un animal en su intento por adaptarse y sobrevivir en las condiciones de su entorno o medio ambiente” (Broom, 1986). Para Dawkins (2016) el animal vive con un adecuado bienestar cuando “está sano y tiene lo que quiere”.
Teniendo en cuenta la bibliografía consultada se puede concluir que el bienestar animal se refiere al estado de felicidad que experimenta un individuo al adaptarse de manera exitosa y positiva ante los cambios del entorno, logrando suplir sus necesidades, lo cual involucra la salud y el confort.
A continuación, se muestra una lista planteada por Blasco (2011) que describe de manera general las principales formas de relación humano-animal a través de la historia:
\begin{enumerate}
	\item Cría de animales en granja para consumir sus productos (leche, huevo, etc.)
\item  Cría y matanza de animales para consumir su carne
\item  Cautiverio de animales fuera de sus ambientes naturales con fines de esparcimiento (zoológicos, circos, parques, etc)
\item  Deporte (caza, pesca)
\item  Experimentación con animales
\item  Animales de compañía
\item  Animales usados para trabajo (guarda, transporte)
\item  Espectáculos con animales amaestrados (acuarios, circos, etc)
\item  Espectáculos que involucran agresividad hacia los animales (tauromaquia, pelea de gallos, pelea de perros, etc)
\item  Tratamiento de plagas (ratas, conejos, insectos, etc)

\end{enumerate}
Desafortunadamente, en la actualidad, la gran demanda de productos de origen animal ha ocasionado el aumento de sistemas intensivos de producción animal que atentan contra el bienestar de los mismos, llegando a ser explotados y considerados como meras máquinas de producción, sin tener en cuenta que son seres que sufren de manera física y emocional, es por ello que hoy en día se han desarrollado una serie de políticas y prácticas para protegerlos.
Según Giménez (2014) la sociedad demanda, cada día más, no solo que los animales domésticos reciban un trato digno y que no proliferen los abandonos y maltratos, sino que los animales se beneficien de una consideración cada vez mayor, que reciban un trato adecuado a su condición de seres vivos sensibles y que la concepción misma del animal como objeto del Derecho alcance una mayor coherencia jurídica. 
Este sentir colectivo se ha venido mostrando, fundamentalmente, en el desarrollo de legislaciones y marcos normativos para funcionar como vía de la protección estatal de los animales y sus posibles derechos, tanto en Europa y Estados Unidos, como en América Latina. 
En el caso de Cuba, fue aprobado en febrero del año 2021 el Decreto-Ley No.31 de Bienestar Animal, lo cual constituye un paso de avance en el empeño de proteger a los animales. Esta ley regula los principios, deberes, reglas y fines respecto al cuidado, la salud y la utilización de los animales para garantizar su bienestar. El Ministerio de Salud Pública del país dispone de programas de vigilancia, prevención y control de enfermedades zoonóticas, enfocados en aquellas que constituyen un riesgo para la salud. Para cumplir con los objetivos de estos programas se realizan diversas acciones como el monitoreo de la ocurrencia de casos, diagnóstico microbiológico, control de focos, atención médica a las personas, vacunación de grupos de riesgo y poblaciones de animales reservorios, educación sanitaria y promoción de salud. (MINSAP, 2022)
