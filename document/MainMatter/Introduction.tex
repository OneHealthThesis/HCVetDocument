
\chapter*{Introducción}\label{chapter:introduction}
\addcontentsline{toc}{chapter}{Introducción}

El trabajo con informaci\'on m\'edica, tanto humana como animal, ha sido a lo largo de la historia un tema sumamente importante y sensible. Debido a la gran cantidad de descubrimientos obtenidos por las ciencias de la salud, con respecto al origen, s\'intomas y tratamientos de los distintos padecimientos, el manejo y la cantidad de datos que es necesario mantener a la hora de establecer un diagn\'ostico correcto es inmensa. Por esta raz\'on han sido creados distintos m\'etodos de preservaci\'on de documentos m\'edicos.

El medio de almacenamiento que mantuvo la mayor popularidad en el anterior siglo fue, por supuesto, el anal\'ogico. Los historiales m\'edicos\footnote{colecci\'on de documentos actualizados por el personal de la salud con respecto a la historia y actualidad de los padecimientos de un individuo} se creaban y actualizaban de forma manuscrita y en ocasiones se almacenaban en algún medio analógico. Este medio mantiene una serie de particularidades que hace dif\'icil su recomendaci\'on. Las principales adversidades que contiene este sistema son su poca facilidad en maleabilidad, replicaci\'on y distribuci\'on. Es correcto mantener la opini\'on de que caracter\'isticas como estas ayudan a garantizar la seguridad, pues en general la información médica tiene un car\'acter privado. Sin embargo, la seguridad no es un bien exclusivo ni absoluto, pues la lucha contra vulnerabilidades se encuentra presente en cualquier tipo de estructura.

Gracias a la informatizaci\'on, varias aplicaciones y dispositivos se encuentran a disposición de pacientes, profesionales, investigadores sanitarios y el p\'ublico general, relacionadas con la salud. En cierta medida, tienen como resultado una mejora de la calidad del tratamiento al respaldar la eficiencia de la asistencia sanitaria profesional, autogestión de pacientes y prevención de enfermedades. Como ejemplos existen medidores de frecuencia card\'iaca y seguidores de niveles de insulina.

Uno de las principales aplicaciones informáticas que surgen a partir de la segunda d\'ecada del siglo XXI, relacionadas con la atenci\'on m\'edica, son los historiales m\'edicos digitales (o historiales clínicos electr\'onicos). Estos programas le permiten a los usuarios mantener la informac\'ion correspondiente almacenada en su dispositivo o en alguna base de datos, permitiendo de esta manera que doctores, e incluso entidades (hospitales, laboratorios, etc...) tengan acceso a ella en caso de ser necesario. Esta estandarizaci\'on present\'o un gran n\'umero de ventajas y permiti\'o la evoluci\'on de muchos procesos sanitarios. 

Existe una enorme diversidad cuando se trata de tipos de historiales m\'edicos digitales existentes. Las caracter\'isticas cambian seg\'un el principal problema que intentan resolver. Por ejemplo, algunas son solo utilizadas por profesionales, otras solo mantienen resultados de pruebas de laboratorios, varias son diseñadas para tratar con una \'unica enfermedad, o solo tratan una rama espec\'ifica de la medicina. Entrando en estas especializaciones, se encuentran aquellas que abordan la veterinaria. Estas se centran en la actualización y conservación de informaci\'on del cuidado de la salud de distintos animales. Cumplen el mismo prop\'osito que las utilizadas en salud humana pero cambiando el tipo de pacientes. La utilidad que presentan es incre\'ible, tanto en el \'ambito dom\'estico como pecuario. La posibilidad de llevar y compartir un registro de los animales ayuda a la prevenci\'on de enfermedades y a la facilidad del tratamiento. 
\newline

Hoy en día el bienestar animal es prioridad de la sociedad a nivel mundial , siendo este definido como un concepto amplio que incluye diversos elementos  contribuyentes a la calidad de vida de un animal, incluidos los referidos en las “cinco necesidades” (libertades) \brackcite{manteca2012bienestar}: necesidad de no sufrir hambre, sed ni desnutrición, necesidad de no experimentar miedo ni angustia
, necesidad de vivir libre de incomodidad física y térmica,
, necesidad de no sufrir de dolor, lesiones y enfermedad,
, y necesidad de expresar patrones normales de comportamiento.


Ha sido definido por la Organización Mundial de Sanidad Animal (OIE) como el término que describe la manera en que los animales se enfrentan con el medio ambiente y que incluye su sanidad, sus percepciones, su estado anímico y otros efectos positivos o negativos que influyen sobre los mecanismos físicos y psíquicos del animal \brackcite{manteca2012bienestar}.

Por otra parte, el biólogo y profesor de bienestar animal Donald Broom lo describe como: “\textit{el equilibrio del estado físico y psicológico de un animal en su intento por adaptarse y sobrevivir en las condiciones de su entorno o medio ambiente}” \brackcite{broom2017animal}. Para Dawkins \brackcite{dawkins2016animal} el animal vive con un adecuado bienestar cuando “\textit{está sano y tiene lo que quiere}”.

 A continuación, se muestra una lista planteada por Blasco \brackcite{blasco2011etica} que describe de manera general las principales formas de relación humano-animal a través de la historia:
\begin{enumerate}
\item Cría de animales en granja para consumir sus productos (leche, huevo, etc.).
\item  Cría y matanza de animales para consumir su carne.
\item  Cautiverio de animales fuera de sus ambientes naturales con fines de esparcimiento (zoológicos, circos, parques, etc.).
\item  Deporte (caza, pesca).
\item  Experimentación con animales.
\item  Animales de compañía.
\item  Animales usados para trabajo (guarda, transporte).
\item  Espectáculos con animales amaestrados (acuarios, circos, etc.).
\item  Espectáculos que involucran agresividad hacia los animales (tauromaquia, pelea de gallos, pelea de perros, etc.).
\item  Tratamiento de plagas (ratas, conejos, insectos, etc.).
\end{enumerate}
En la actualidad, desafortunadamente, la gran demanda de productos ha ocasionado el aumento de sistemas intensivos de producción animal que atentan contra el bienestar de los mismos. Ciertas especies animales son explotadas y consideradas como meras máquinas de producción, sin tener en cuenta que son seres que sufren de manera física y emocional. Es por ello que hoy en día se han desarrollado una serie de políticas y prácticas para protegerlos. 

Según Giménez \brackcite{gimenez2014animales} la sociedad demanda, cada día más, no solo que los animales domésticos reciban un trato digno y que no proliferen los abandonos y maltratos, sino que se beneficien de una consideración cada vez mayor, que reciban un trato adecuado a su condición de seres vivos sensibles y que la concepción misma del animal como objeto del Derecho alcance una mayor coherencia jurídica. Este sentir colectivo se ha venido mostrando, fundamentalmente, en el desarrollo de legislaciones y marcos normativos para funcionar como vía de la protección estatal de los animales y sus posibles derechos, tanto en Europa y Estados Unidos, como en América Latina.  

En el caso de Cuba, fue aprobado en febrero del año 2021 el Decreto-Ley No.31 de Bienestar Animal, lo cual constituye un paso de avance en el empeño de proteger a los animales. Esta ley regula los principios, deberes, reglas y fines respecto al cuidado, la salud y la utilización de estos individuos para garantizar su bienestar. El Ministerio de Salud Pública del país (MINSAP) dispone de programas de vigilancia, prevención y control de enfermedades zoonóticas, enfocados en aquellas que constituyen un riesgo para la salud. Para cumplir con los objetivos de estos programas se realizan diversas acciones como el monitoreo de la ocurrencia de casos, diagnóstico microbiológico, control de focos, atención médica a las personas, vacunación de grupos de riesgo y poblaciones de animales, educación sanitaria y promoción de salud. \brackcite{Minsap2022Feb}



La promulgación sistemática de leyes que protegen a estos seres vivos y prohíben prácticas violentas e innecesarias en contra de su vida y dignidad, evidencia una tendencia al reconocimiento y la protección normativa del derecho al bienestar de los animales \brackcite{jarrin2021diseno}. Sin lugar a dudas, las personas se hallan cada vez más preocupadas por el cuidado de los mismos; lo cual ha conllevado a la generación de nuevas tecnologías para optimar los servicios de atención a los animales. Es en este punto donde se inscribe el presente trabajo mediante el cual se intenta desarrollar una herramienta para conservar de forma segura y conveniente el historial médico de animales afectivos. Esta herramienta, en forma de aplicación móvil con respaldo de datos en un servidor, será de ayuda tanto a propietarios de mascotas como a médicos veterinarios en la consecución de un mayor bienestar de los animales bajo su cuidado. 


\subsection*{Motivaci\'on}

Desde hace una d\'ecada los historiales clínicos electr\'onicos forman parte del sistema natural de atenci\'on en gran parte de los sistemas de salud del mundo. La conveniencia y agilidad que brindan facilitan el proceso de atenci\'on al paciente, as\'i como el trabajo que realiza el profesional. \textbf{Las grandes capacidades que pueden ser explotadas de este campo es justificaci\'on suficiente para la realizaci\'on de esta investigaci\'on}. Durante todo este proceso hemos mantenido como motivaci\'on principal el avance que estos resultados podr\'ian mostrar al sistema de salud cubano. Teniendo en cuenta los efectos que han mantenido otras investigaciones y la aplicaci\'on pr\'actica de ellas, los veterinarios cubanos podr\'ian llegar a aumentar en gran medida el \'indole y la calidad de los servicios que ofrecen.

\subsection*{Formulaci\'on del problema}

A los dueños y cuidadores de animales domesticables les sería conveniente tener una herramienta que les permita gestionar los datos clínicos históricos de la condición de salud y los servicios que han recibido los animales a su cargo. También al sector de los profesionales que ofrecen servicios a estos animales le resultaría útil contar con un medio ordenado y de fácil acceso de datos sobre los animales bajo su cuidado. Dicha herramienta estaría orientada a agilizar la prestación del servicio dado y a mejorar la calidad de este mediante el acceso a información recurrente sobre los animales atendidos. Las principales funcionalidades del producto propuesto son:
\begin{itemize}
\item Almacenar información de forma segura sobre las características básicas del animal bajo el cuidado del usuario.
\item Proveer acceso a los profesionales a información histórica sobre el cuidado de los animales para no repetir errores cometidos anteriormente y agilizar la aplicación de sus servicios.
\item Crear de forma rápida nueva información sobre los animales.
\item Permitir el acceso a datos ,controlado por el dueño de un animal, a otros usuarios con el fin de delegar el cuidado del animal.
\item Sincronizar nuevos datos creados por varios usuarios con un mismo animal bajo su cuidado.
\end{itemize}

\subsection*{Hip\'otesis}
Utilizando el lenguaje de programación C\#, el framework .NET 6 y las herramientas construidas sobre este, se podrá implementar un sistema multiplataforma capaz de realizar las funcionalidades antes mencionadas.

Trabajando sobre la plataforma Flutter a trav\'es de Dart, es posible crear un interfaz de usuario funcional, as\'i como la posibilidad de implementar las capacidades de transmisi\'on de informaci\'on mediante una red local.

\subsection*{Objetivos}

En este trabajo se establece como objetivo el diseño y desarrollo de un sistema multiplataforma dedicado a la gestión de información (Historia Clínica) de animales domesticables. Se propone el uso de una base de datos relacional y un sistema alojado en un servidor que permita el acceso a los datos y exponga una API de fácil uso para que servicios clientes accedan a ella mediante el protocolo HTTPS.

\subsection*{Tareas}

\begin{itemize}
\item Investigar sobre el ciclo de desarrollo de software eficiente y sostenible y sobre arquitecturas de software que permitan implementar un sistema de fácil mantenimiento y extensión.
\item Analizar y probar bibliotecas de .NET 6 que permitan el rápido desarrollo del producto deseado.
\item Implementar una API que permita que cualquier cliente autorizado tenga acceso a sus datos a través de ella.
\item Despliegue del software de forma pública y gratuita para una etapa de prueba del producto.
\end{itemize}

\subsection*{Organizaci\'on de la tesis}

El Trabajo de Diploma consta de cinco capítulos, seguidos de conclusiones, recomendaciones y referencias bibliográficas.
\begin{itemize}
\item Capítulo 1:  “Bienestar animal e Historia Clínica Veterinaria”, en este capítulo se explica el desarrollo social alcanzado a nivel internacional y nacional sobre el bienestar animal. Se exponen conceptos que justifican la necesidad de elevar la conciencia sobre el tema ya que junto al ser humano los animales conforma el ecosistema natural y su cuidado tiene un impacto directo en la calidad de vida del planeta. Se plantea la contribución que el resultado del presente trabajo prevé en el ámbito del bienestar animal en nuestro país. 

\item Capítulo 2: “Estado del Arte”, se expone un estudio de varias iniciativas similares a la presente propuesta. Se analiza el origen de los softwares usados actualmente, sus características y su impacto social. Además se expone brevemente las características del producto a desarrollar.

\item Capítulo 3: “Producto”, análisis profundo del producto y presentación de una versión lista para etapa de prueba y de análisis de aceptación de usuarios. 

\item Capítulo 4: “Desarrollo”, explicación del ciclo de desarrollo y análisis de las herramientas y tecnologías seleccionadas para la implementación de la solución.

\item Capítulo 5: “Despliegue”, se revisará la etapa final del desarrollo del producto, así como las tecnologías usadas para lograr su despliegue y la importancia de dichas herramientas para la correcta ejecución de esta etapa de desarrollo.

\end{itemize}
