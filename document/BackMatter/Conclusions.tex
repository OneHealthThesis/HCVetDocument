\begin{conclusions}
    Luego del desarrollo e implementación del proyecto; y la obtención de resultados, 
    se evidenció la importancia  que presenta en la actualidad el cuidado y 
    protección de los animales y la necesidad de realizar aplicaciones informáticas para agilizar y facilitar la 
    prestación de los servicios veterinarios. 
    
    El objetivo de este trabajo fué la creación de una API de la cual va a consumir la App móvil HCVet, de la que pudimos ver sus casos de uso, se hizo un análisis teórico computacional exponiendo la arquitectura utilizada y el modelo de datos. También se vieron detalles de implementación, el porqué de las herramientas y lenguajes utilizados.
    
    De manera general se cumplió el objetivo de la hipótesis planteada, cumpliendo con las funcionalidades propuestas de las cuales se muestran algunas a continuación:
    \begin{itemize}
    	\item Almacenar información de forma segura sobre las características básicas del animal bajo el cuidado del usuario.
    	\item Proveer acceso a los profesionales a información histórica sobre el cuidado de los animales para no repetir errores cometidos anteriormente y agilizar la aplicación de sus servicios.
    	\item Crear de forma rápida nueva información sobre los animales.
    	\item Permitir el acceso a datos ,controlado por el dueño de un animal, a otros usuarios con el fin de delegar el cuidado del animal.
    
    \end{itemize}
    

\end{conclusions}
