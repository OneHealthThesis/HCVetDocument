
\chapter{Estado del Arte}\label{chapter:state-of-the-art}
A través de la historia de la humanidad, ha sido clara la relación de dependencia con el mundo natural para sobrevivir. Cada época, de acuerdo con las capacidades del ser humano, su grado de desarrollo científico y tecnológico y su entendimiento del mundo, ha gestado una versión de veterinarios que generaron las condiciones para que el sector agropecuario fuera capaz de proveer los medios suficientes y necesarios para fomentar el avance de la sociedad al fortalecer su estructura y aumentar su complejidad. 

En la actualidad, cuando se piensa en el ejercicio profesional de las ciencias veterinarias, es común hacer relaciones conceptuales con asuntos de salud y enfermedad de animales domésticos, lo que conduce a aspectos particulares de las disciplinas como la patología, la parasitología, la clínica, la cirugía, etc. Relaciones también con la agricultura, lo que le da un contexto histórico relacionado con la satisfacción de necesidades básicas de los seres humanos, al tiempo que le confiere una responsabilidad desde la perspectiva de una disciplina importante para la conservación de la fauna silvestre y la diversidad. 
\brackcite{vela2012medicina}

Las condiciones existentes hoy día constituyen un gran desafío para los profesionales de esta disciplina.  El cambio climático, el aumento de la población, la afectación al medio ambiente, la intensificación de la producción pecuaria y agrícola, la alteración de los ecosistemas, están afectando la dinámica de la salud animal y humana, razón por la cual han incrementado los estudios, así como las estrategias y planes para mitigarlo. 

“En esta era de la globalización, el desarrollo y el crecimiento de muchos países, así como la prevención y la lucha contra las principales catástrofes biológicas, dependen del resultado de sus políticas y economías en materia agrícola, alimentaria y de sanidad animal, lo que, a su vez, está directamente relacionado con las actividades y la calidad de los Servicios Veterinarios nacionales”. (Organización Mundial de Sanidad Animal, 2013) 

Ante esta situación, en los últimos años se ha consolidado la necesidad de adoptar un enfoque interdisciplinario y multisectorial en el manejo de la salud de los seres humanos, los animales y los ecosistemas (Zunino, 2018). 

En este sentido, la Organización para la Agricultura la Alimentación (FAO), la Organización Mundial de Sanidad Animal (OIE) y la Organización Mundial de Salud (OMS) han desarrollado una nota conceptual tripartita que sienta una orientación estratégica y propone una base de colaboración internacional a largo plazo con objeto de coordinar las actividades a nivel mundial para superar los riesgos para la salud en la interfaz entre humanos, animales y ecosistemas. Esta colaboración es conocida como el enfoque «Una sola salud». (OIE, 2015) 
\section{Softwares Existentes}
A continuación, se presentan estudios previos o antecedentes investigativos, como referentes para este proyecto, en los cuales se establecen la utilización de las aplicaciones móviles para la prestación de servicios médicos veterinarios. 
\subsection{Guau}
\brackcite{jarrin2021diseno}
Un estudio que fue de vital importancia para la temática de estudio lleva por nombre \textbf{Aplicación Móvil para Centralizar Servicios de Mascotas}, elaborado por Maribel Boggio en el año 2017, como objetivo central se pretende construir una aplicación móvil para centralizar servicios de mascotas, su problema se justifica en que en los últimos años las mascotas han ido adquiriendo mayor protagonismo en la vida de los humanos, formando parte importante del núcleo familiar y recibiendo los mismos cuidados que cualquiera de sus miembros. Las familias que tienen una mascota se preocupan cada vez más por brindarle los mejores cuidados que les sea posible; pero surge un inconveniente, en muchas ocasiones por más que deseen brindarles los mejores cuidados, no saben dónde realizarlos, si los lugares a los cuales van tienen una buena o mala reputación, buscan servicios y muchas veces no los encuentran, no saben que hay ciertos procedimientos, tratamientos que pueden beneficiar a sus mascotas, inclusive ofertas de productos que pudieran adquirir. En términos metodológicos se realizaron varios estudios de mercado donde se evidenció una falta de servicios para las mascotas. 

Como resultados se obtuvo que la implementación de la Aplicación \textbf{Guau} es un proyecto factible y financieramente viable con una TIR de 29\%. Esto se sustenta en el análisis económico – financiero. El horizonte de estudio fue definido en 4 años debido a que, si bien es una solución tecnológica y todo hace indicar que continuará en ascenso, no podemos perder de vista la rapidez y evolución de este mercado. La estrategia de la empresa será de diferenciación, en el mercado no existe algo similar y el contacto con los clientes será personalizado de acuerdo a sus necesidades. Hoy en día en el Perú existe una clara preferencia por perros como mascota, por ello el desarrollo de la aplicación se centrará en canes. No debemos perder de vista al gato como mascota, el cual tiene una tendencia de crecimiento importante, aunque todavía lejos de los perros. 
\subsection{PetSoft}
\brackcite{petsoft}
Pet Soft es una plataforma para veterinarias y dueños de mascotas que busca tener una red de veterinarias unidas, las cuales compartirán la información de los pacientes, de modo que los usuarios puedan utilizar cualquier centro de servicio veterinario asociado a la red, sin tener que abrir una nueva historia clínica cada vez que cambien de lugar de residencia o que tengan una urgencia médica y no puedan llevar  su amigo peludo al lugar que frecuentan constantemente.

Este Software tiene una interfaz amable, de fácil uso y además cuenta con tecnología de punta que permite ingresar toda la información requerida y necesaria de cada mascota y visualizarla rápidamente.  Como valor agregado, cuenta con una App móvil, que es un canal directo de comunicación con los clientes y que hace que todo se gestione de forma rápida, ágil y sencilla, por medio del sistema de agendamiento virtual y el mapa de la red de veterinarias que se encuentran a lo largo de la ciudad, donde los usuarios podrán ubicar el punto de servicio más cercano.
\newline
\newline


\textbf{Características de PetSoft}


\begin{enumerate}
\item	\textbf{Agenda inteligente}: Los usuarios podrán agendar citas de acuerdo a su disponibilidad de tiempo, escogiendo entre el abanico de posibilidades de servicio  que brindan los centros veterinarios, tales como peluquería, medicina, venta de productos, entre otras especialidades.  Esta agenda también envía notificaciones avisando la hora y día de la cita, tanto al centro médico, como al dueño de la mascota.
	
	
	
 \item \textbf{Mapa de ubicación de veterinarias}: Los clientes podrán llevar a sus mascotas a lugar que les quede más cerca de donde se encuentren.
	
	
\item \textbf{Historias clínicas compartidas}: La información de cada mascota, queda almacenada en la nube y puede ser utilizada por todas las veterinarias asociadas a la red PetSoft.
	
	
\item \textbf{Carné virtual}: El carné se lleva en el celular, ya que se visualiza a través de la App.
\end{enumerate}



\subsection{Sistema de información para el control de expedientes clínicos para médicos veterinarios}

Sistema capaz de gestionar el control de historias 
clínicas veterinarias, además de la utilización con fines educativos e informativo ya que 
cuenta con las funciones de la publicación de contenidos referentes a la carrera , tales 
como: artículos, imágenes, foros, blog, libros, etc. Además facilita el trabajo debido a la 
función de guardar y exportar los reportes. \brackcite{thesisSistemaInf}
\newline


\textbf{{\large Conclusiones del Capítulo:}}
\begin{enumerate}
	\item Los desafíos que enfrentan los servicios sanitarios hoy día ha 
	conllevado al establecimiento de un enfoque integrador y 
	multidisciplinario que involucra la relación entre los humanos, 
	animales y ecosistemas.
	\item Existen numerosos estudios sobre el desarrollo de aplicaciones 
	veterinarias que fueron tomados como base para el desarrollo de este 
	proyecto, como son: la aplicación Guau y la plataforma PetSoft.
	\item Dentro de las funcionalidades de estos softwares se hallan: mayor 
	personalización del servicio a mascotas y sus dueños, conectar las 
	redes de veterinaria de modo que puedan atender a sus animales en 
	diferentes centros sin necesidad de abrir una nueva historia clínica, 
	brindando disponibilidad de tiempo y lugar; mayor accesibilidad y 
	flexibilidad, entre otras.
\end{enumerate}



