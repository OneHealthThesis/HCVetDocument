\chapter{Producto}\label{chapter:proposal}
HCVet es una App movil para gestionar y administrar histórias clínicas de mascotas, donde los usuarios pueden tener registros de cada consulta que se ha hecho su mascotas,saber sus padecimientos y antecedentes de forma rápida y precisa, siendo esto de mucha ayuda a la hora de atender a su mascota en las visitas al veterinario,y tambien para tener conciencia de los cuidados y la atención que se le debe dar.

Se tiene un servidor para manejar la sincronizacion entre los clientes, la primera vez que una persona va a usar la app debe registrarse, por tanto debe conectarse al servidor, este le asignara dos espacios para registrar dos mascotas, en caso de querer más debera realizar una suscripción.

La app cuenta con varias vistas para distintos tipos de consulta que se pueda hacer su mascota como patología, radiología, una visita médica regular, una sirugía, ect. Cada una cuenta con una serie de datos que se pedirán para llenar segun el tipo de consulta. Los usuarios también podrar asignar a otros como \textbf{encargados} de sus mascotas, para que estos también puedan registrar consultas sobre  las mascotas de otros dueños, por ejemplo un familiar o amigo.

Una de las cosas mas particularas que tiene este software es que está pensado para que casi todas sus funcionalidas se puedan hacer fuera de líena (offline), o sea, sin estar conectado a internet. Se pensó de esta forma para evitar la total dependencia de un servidor, ya que a algunas personas les puede ser un poco difícil el acceso a la red. Debido a esto se tuvieron que hacer algunas modificaciones en la estcutura del proyecto, para que el servidor pudiera hacer un proceso de sincronizacion, correcto y eficiente.Los datos de todos los usuarios estarán en el servidor, pero tambien cada uno tendrá sus datos particulares en una base de datos local en su móvil. El proceso de sincronización cuenta con las siguientes funcionalidades, que estas si requieren de conexión a internet :
\begin{itemize}
	\item Subir datos al servidor :
	
	Los nuevos datos generados se subirán al servidor. Esto se realizo para tener una centralizacion de los datos, y de alguna forma una copia de seguridad de los mismos, en caso de perdida. También sirve para el proceso de sincronización en usuarios que tengan masctas en común, al tener esos datos en el servidor, proveerá esos datos a los usuarios que le sean requeridos.
	
	\item Bajar datos del Servidor :
	
	Todos los datos que hayan sido subidos al servidor, de todas las mascotas que el usuario sea dueño serán agregados a la base de datos local. Está funcion mantiene actualizados los datos de cada usuario, por ahora solo agrega daatos, en un futuro se pudiara mejorar esta acción para que también actualice y elimine datos, que este proceso puede ser bastante complejo. 
	
	\item Asignar encargado :
	
	Se le notificará al servidor que un usuario ha sido asignado como encargado de una nueva mascota. Esta funcion es necesaria, para que el servidor pueda proveer de los nuevos registros tanto a los dueños como a los encargados.
	
		\item Dessignar encargado :
	
	Se le notificará al servidor que un usuario ha dejado de ser  encargado de una mascota. A la vez que un cliente deje de tener relacion con la mascota, se deberá tener encuenta para que no lleguen datos innecesarios a la hora de sincronizar   
\end{itemize}

  
 

