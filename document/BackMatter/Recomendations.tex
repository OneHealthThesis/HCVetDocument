\begin{recomendations}
	Aunque se cumplió con todos los objetivos planteados al inicio del trabajo, el software implementado puede mejorarse y se pueden agregar nuevas funcionalidades, por ello, se propone:
	
	\begin{itemize}
   \item Hacer convenios con las diferentes autoridades sanitarias para que conozcan
   de la app y se puedan establecer estadísticas más precisas sobre la situación 
   de las mascotas y animales en general, para darle amplitud al espectro de 
   servicios de la app y al propio tiempo, para que se mantengan a la vanguardia 
   de las últimas tecnologías referentes al campo veterinario.
   \item Identificar el estado inicial de los procesos y actividades de una clínica veterinaria 
   permitirá tener una visión completa y objetiva de la situación y conocer cuáles son 
   los problemas más críticos para poder priorizarlos. 
  \item Mejorar el proceso de sincronización, permitiendo que se pueda modificar y eliminar datos, y estas modificaciones se vean reflejadas en todos los usuarios relacionados con estos.
    \item Continuar con el diseño e implementación de módulos para agregar nuevas funcionalidades.
    \item Se debe dar una capacitación constante al personal de la clínica veterinaria a 
   implementar el uso de esta app, de forma que puedan, gradualmente, comprender 
   las opciones y procesos para su correcto empleo.
   	\end{itemize}
\end{recomendations}
